%\setpartpreamble[uc][0.7\textwidth]{\vspace*{0.4cm}
%Our model can be separated into two modules. The first one deals with the cluster dynamics. It tells us how the gas in the cluster behaves, when new stars form etc. The second module tells us what yields we expect from the clusters described in the previous module. Together they can follow the chemical enrichment in a given star forming cloud in a self consistent way.
%}
%\part{The Model}
\addcontentsline{toc}{part}{model}

%--------------------------------------------------------------
% CHAPTER: Population synthesis
%--------------------------------------------------------------

\chapter{Introduction}\label{sec:introduction}
We revisit a paper by \textcite{kreckel2017}. In this paper we use data from the \gls{muse} instrument of the \gls{vlt} to identify \gls{pn} and use the \gls{pnlf} to measure their distance. \Glspl{pn} are a reliable distance measure


