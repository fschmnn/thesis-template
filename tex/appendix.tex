%\part{Appendix}
\appendix

\chapter{Parameters and results}\label{sec:appendixA}

Hereinafter we list additional information like the cloud parameters and the abundance spreads. \cref{tbl:initial_fiducial} shows the parameters for our fiducial model. Since most models use the same \textsc{warpfield} output this table is valid for all, but the model with an higher ambient density. The differences are between the most enriched/depleted population and the initial abundances e.g.~$\Delta \abundance{Fe}{H} = \abundance{Fe}{H}_\text{max} - \abundance{Fe}{H}_\text{FG}$. For helium and CNO, the difference in mass fraction is listed.

% wraper around pgfplotstabletypeset with extra options
\begin{table}[H]
\centering\small
\caption[Initial cloud parameters]{Initial cloud parameter for a selection of models that were used to showcase the results. $M_\text{cl}$ is the initial cloud mass, $\SFE$ the star formation efficiency, $n_\text{cl}$ the cloud density, $N_\text{pop}$ is the number of populations that formed and $M_\text{cluster}$ is the final mass of the cluster.}
\insertdatatable{data/fiducial.txt}
\label{tbl:initial_fiducial}
\end{table}

\begin{table}[t]
\centering\small
\caption[Enrichment for fiducial model]{Abundance spreads for our fiducial model.}
\insertresulttable{data/fiducial.txt}
\label{tbl:data_fiducial}
\end{table}


\begin{figure}[b]
\centering
\input{trecollapse.pgf}
\caption[Re\-/collapse time and \textsc{warpfield} parameters]{Time for the first re\-/collapse as a function of different \textsc{warpfield} parameters. $t_\text{re-collapse}$ is defined as the time between the furthest extension of the shell and the formation of the next cluster.}
\label{fig:trecollapse}
\end{figure}