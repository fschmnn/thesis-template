\chapter{Photometry}

\section{Growth curve analysis}

\subsection{Gaussian}

A shape that is commonly used for the PSF is that of a 2D gaussian (we assume  variance of $\sigma_x^2 = \sigma_y^2 = \sigma^2$). If we center the peak at the origin the \gls{psf} is described by
\begin{align}
f(x,y) = A \exp\left(-\frac{x^2+y^2}{2\sigma^2}\right)
\end{align}
with some amplitude $A$. We can rewrite this in polar coordinates as 
\begin{align}
f(r,\phi) = A \exp\left(-\frac{r^2}{2\sigma^2}\right)
\end{align}
The light inside an aperture of radius $P(R)$ is given by the integral
\begin{align}
P(R) = \int_0^{2\pi} \int_0^R f(r,\phi) \mathrm{d} \phi r \mathrm{d} r = 2\pi \sigma^2 A \left(1-\exp \left(-\frac{R^2}{2\sigma^2}\right) \right) 
\end{align}
We are interested in the ratio $p(R) = P(R) / P(\infty)$. If we use the relation between the standard deviation and the $\fwhm$ of a Gaussian $\fwhm = \sigma  2\sqrt{2\ln2}$, we can write
\begin{align}
p(R) = 1-\exp\left(- \frac{4 \ln 2 \cdot R^2}{\fwhm^2} \right)
\end{align}

\subsection{Moffat}

The measured FWHM are systematically larger than the reported values. A possible cause is that the shape of the PSF is not a perfect Gaussian, but rather described by a Moffat. This distribution is larger towards the wings and fitting a Gaussian to such a shape should result in a larger $\fwhm$

\begin{align}
f(R;\alpha,\gamma) = A \left[1 + \left(\frac{R}{\gamma}\right)^2 \right]^{- \alpha}
\end{align}


**Note**: this nomenclature follows `astropy` and contradicts the commonly used scheme which uses $\gamma=\alpha$ and $\alpha=\beta$.

The Full Width Half Maximum of this function is given by
\begin{align}
\fwhm = 2\gamma \sqrt{2^{1/\alpha}-1}
\end{align}


like we did for the Gaussian we can calculate the amount of flux within a radius R as 

\begin{align}
P(R) = \int_0^{2\pi} \int_0^R f(r,\phi) \mathrm{d} \phi r \mathrm{d} r = 2\pi \int_0^R A \left[1 + \left(\frac{r}{\gamma}\right)^2 \right]^{- \alpha} r \mathrm{d} r
\end{align}


to solve this we substitute $u=1+\left(\frac{r}{\gamma} \right)^2 $ with $\frac{\mathrm{d} u}{\mathrm{d} r} = \frac{2r}{\gamma^2}$. 


\begin{align}
P(R) = A \frac{\gamma^2 \left(1 + \left( \frac{R}{\gamma} \right)^2 \right)}{2(1-\alpha)\left( 1+\left( \frac{R}{\gamma}\right)^2\right)^\alpha} - A\frac{\gamma^2}{2(1-\alpha)}
\end{align}


again we are interested in the ratio $p(R) = P(R) / P(\infty)$. If we assume that $\alpha>1$, the first term will be $0$ for $R\rightarrow \infty$ and so we end up with 

\begin{align}
p(r) = \left[ 1+\left( \frac{R}{\gamma}\right)^2\right]^{1-\alpha} - 1
\end{align}

