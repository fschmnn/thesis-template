%--------------------------------------------------------------
% Introduction Part
%--------------------------------------------------------------

\setpartpreamble[uc][0.7\textwidth]{\vspace*{0.4cm}
If parts are used, text for the preamble goes here
}
%\part{Introduction}
\addcontentsline{toc}{part}{introduction}

\chapter{Overview}\label{sec:overview}

In \cref{sec:main} we talk a little bit about some subjects to showcase this template and in \cref{sec:summary} we summarize everything.

\section{Abbreviations}


\section{bibliography}


\section{}


%--------------------------------------------------------------
% Main Part
%--------------------------------------------------------------

\setpartpreamble[uc][0.7\textwidth]{\vspace*{0.4cm}
You can use it to summarize what follows in the included chapters.
}
%\part{Main}
\addcontentsline{toc}{part}{main}

\chapter{Showcase}\label{sec:main}

We should talk about something. Therefore we reference the old Astronomy masters like \citet{Newton+1687} or \citet{Hubble+1929}. 

These days, the agencies like \gls{nasa} or \gls{esa} provide us formidable telescopes like \gls{hst}, \gls{jwst} or the \gls{vlt}. The \gls{vlt} is home to the \gls{ifu} instrument \gls{muse} which is extremly powerful to study the \gls{ism}.

\begin{table*}
    \centering
    \caption[Example table]{This table showcases how siunitx can be used in a table.}
    \begin{tabular}{
        l
        S[table-format=1.2]
        S[table-format=2.2]@{\,\( \pm \)\,}
        S[table-format=1.2]
    }
    \toprule
    {Name} & {Value1} & \multicolumn{2}{c}{Value 2}  \\
     & \multicolumn{1}{c}{Unit 1} & \multicolumn{2}{c}{Unit 2}  \\\midrule
    Value 1 & 3.14 & 13.34 & 0.32 \\
    Value 2 & 6.28 & 9.34 & 0.32 \\

    \bottomrule
    \end{tabular}
    \label{tbl:example}
    \end{table*}
    

%--------------------------------------------------------------
% Conclusion Part
%--------------------------------------------------------------

\setpartpreamble[uc][0.7\textwidth]{\vspace*{0.4cm}
If parts are used, text for the preamble goes here
}
%\part{Conclusion}
\addcontentsline{toc}{part}{conclusion}

\chapter{Summary}\label{sec:summary}

This is hopefully useful when writing a paper in an scientific field (especially Astronomy). 

